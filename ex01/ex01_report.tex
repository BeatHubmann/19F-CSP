\documentclass[11pt,a4paper]{article}

% These are extra packages that you might need for writing the equations:
\usepackage{amsmath}
\usepackage{amsfonts}
\usepackage{amssymb}
\usepackage{booktabs}
\usepackage{hyperref}
\usepackage{listings}
\usepackage{xcolor}
\lstset {language=C++,
		 basicstyle=\ttfamily,
         keywordstyle=\color{blue}\ttfamily,
         stringstyle=\color{red}\ttfamily,
         commentstyle=\color{purple}\ttfamily,
         morecomment=[l][\color{magenta}]{\#},
       	 basicstyle=\tiny}

% You need the following package in order to include figures in your report:
\usepackage{graphicx}

% With this package you can set the size of the margins manually:
\usepackage[left=2cm,right=2cm,top=2cm,bottom=2cm]{geometry}


\begin{document}

% Enter the exercise number, your name and date here:
\noindent\parbox{\linewidth}{
 \parbox{.25\linewidth}{ \large CSP, Exercise 01 }\hfill
 \parbox{.5\linewidth}{\begin{center} \large Beat Hubmann \end{center}}\hfill
 \parbox{.2\linewidth}{\begin{flushright} \large Mar 07, 2019 \end{flushright}}
}
\noindent\rule{\linewidth}{2pt}


\section{Introduction}

The single spin flip Metropolis algorithm was implemented as an example Monte Carlo method to simulate
the 3D Ising model. Energy, Magnetization as well as heat caoacity and susceptibility of the system were measured at different temperatures for two different
system sizes.

\section{Algorithm Description}
The algorithm is implemented as originally published~\cite{metropolis} and as described in the lecture notes~\cite{herrmann}.
Specific to the 3D Ising model, a new configuration is generated by randomly flipping the spin of a lattice element
and then calculating the system's post-change energy.
The dimensionless energy $E$ and magnetization $M$ are calculated as indicated in equations~\ref{eqn:1}~and~\ref{eqn:2}, where
$S_i \in \{-1, +1\}$ stands for 'spin down' respectively 'spin up' at site $i$ and $N$~stands for the total number of sites in
the 3D lattice under consideration.
The implemented version of the algorithm allows for several system sweeps after each increasing temperature step,
after which every tenth value for $E$ and $M$ is measured during further system sweeps and then averaged to obtain
the respective site values. By doing so, any correlation between measurements should be minimized.\\
Heat capacity $C$ and magnetic susceptibility $\chi$ are calculated using the fluctuation-dissipation theorem as described in~\cite{boettcher}.

\begin{equation}
	E = -J \cdot \sum_{<i, j>}S_i S_j
\label{eqn:1}
\end{equation}


\begin{equation}
	M = \frac{1}{N}\sum_{i=1}^{N}S_i
\label{eqn:2}
\end{equation}


\section{Results}

The program was implemented as described above and submitted with this report. 
For all experiments, the coupling constant $J$ was fixed to the simplest ferromagnetic value of $J=1.0$. The temperature $T$
was varied in the range from $1$ to $7$ $J/k_B$. No cold starts were used.
The experiment was run twice for 3D Ising lattice side lengths $L \in \{30, 60\}$. The results for dimensionless energy $E$ and magnetization $M$
 for the two lattice sizes are shown in figure~\ref{fig:1} while the results for heat capacity $C$ and magnetic susceptibility $\chi$ are shown in figures~\ref{fig:2} and~\ref{fig:3}.
The site values between the two different system sizes agree while the larger system shows smoother behaviour as one would expect.
By implementing a lookup table for the changes in energy during the Monte Carlo step, the run time of the simulation could be kept
in the order of magnitude of minutes.

\begin{figure}[ht]
\begin{center}
\includegraphics[scale=1.4]{figure1.eps} 
\end{center}
\caption{Value of magnetization $M$ and energy $E$ at different temperatures $T$ according to Metropolis MC with $J=1$ on lattices of side length $L\in\{30,60\}$.}
\label{fig:1}
\end{figure}

\begin{figure}[ht]
\begin{center}
\includegraphics[scale=1.4]{figure2.eps} 
\end{center}
\caption{Value of heat capacity $C$ and magnetic susceptibility $\chi$ at different temperatures $T$ according to Metropolis MC with $J=1$ on lattices of side length $L=30$.}
\label{fig:2}
\end{figure}

\begin{figure}[ht]
\begin{center}
\includegraphics[scale=1.4]{figure3.eps} 
\end{center}
\caption{Value of heat capacity $C$ and magnetic susceptibility $\chi$ at different temperatures $T$ according to Metropolis MC with $J=1$ on lattices of side length $L=60$.}
\label{fig:3}
\end{figure}

\section{Discussion}
As expected, the magnetization $M$ drops off steeply at the critical temperature $T_c \approx 4.51 \quad [J/k_B]$.
Also in line width expectations, the energy increases with temperature with the steepest ascent also around $T_c$.\\
While their jumps happen around $T_c$ as they should, I'm not quite happy with the shapes of the curves of $C$ and $\chi$.
As I can't seem to find any calculation errors, I'd next look into finer temperature resolutions and larger grids.


\pagebreak
\begin{thebibliography}{99}


\bibitem{metropolis}
Metropolis, N.,
Rosenbluth, A.W.,
Rosenbluth, M.N.,
Teller, A.H.,
Teller, E.\\
\emph{Equations of State Calculations by Fast Computing Machines},\\
Journal of Chemical Physics. 21 (6): 1087,\\
1953.


\bibitem{herrmann}
	Herrmann, H. J.,
	Singer, H. M.,
	Mueller L.,
	Buchmann, M.-A.,\\
	\emph{Introduction to Computational Physics - Lecture Notes},\\
	ETH Zurich,\\
	2017.

\bibitem{boettcher}
	Boettcher, L.,\\
	\emph{Computational Statistical Physics - Lecture Notes},\\
	ETH Zurich,\\
	2019.

\end{thebibliography}

\end{document}